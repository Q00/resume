%!TEX TS-program = xelatex
%!TEX encoding = UTF-8 Unicode
% Awesome CV LaTeX Template for CV/Resume
%
% This template has been downloaded from:
% https://github.com/posquit0/Awesome-CV
%
% Author:
% Claud D. Park <posquit0.bj@gmail.com>
% http://www.posquit0.com
%
% Template license:
% CC BY-SA 4.0 (https://creativecommons.org/licenses/by-sa/4.0/)
%


%-------------------------------------------------------------------------------
% CONFIGURATIONS
%-------------------------------------------------------------------------------
% A4 paper size by default, use 'letterpaper' for US letter
\documentclass[11pt, a4paper]{awesome-cv}

% Configure page margins with geometry
\geometry{left=1.4cm, top=.8cm, right=1.4cm, bottom=1.8cm, footskip=.5cm}

% Specify the location of the included fonts
\fontdir[fonts/]
\usepackage{kotex}
\usepackage{hyperref}
\hypersetup{
    colorlinks=true,
    linkcolor=blue,
    filecolor=magenta,      
    urlcolor=cyan,
}

% Color for highlights
% Awesome Colors: awesome-emerald, awesome-skyblue, awesome-red, awesome-pink, awesome-orange
%                 awesome-nephritis, awesome-concrete, awesome-darknight
\colorlet{awesome}{awesome-red}
% Uncomment if you would like to specify your own color
% \definecolor{awesome}{HTML}{CA63A8}

% Colors for text
% Uncomment if you would like to specify your own color
% \definecolor{darktext}{HTML}{414141}
% \definecolor{text}{HTML}{333333}
% \definecolor{graytext}{HTML}{5D5D5D}
% \definecolor{lighttext}{HTML}{999999}

% Set false if you don't want to highlight section with awesome color
\setbool{acvSectionColorHighlight}{true}

% If you would like to change the social information separator from a pipe (|) to something else
\renewcommand{\acvHeaderSocialSep}{\quad\textbar\quad}


%-------------------------------------------------------------------------------
%	PERSONAL INFORMATION
%	Comment any of the lines below if they are not required
%-------------------------------------------------------------------------------
% Available options: circle|rectangle,edge/noedge,left/right
% \photo[rectangle,edge,right]{./examples/profile}
\name{JaeGyu}{Lee}
\position{Software Engineer}
\address{445, Dobong-ro, Dobong-gu, Seoul, Republic of Korea}

\mobile{(+82) 10-9937-5774}
\email{jqyu.lee@gmail.com}
% \homepage{}
\github{github.com/q00}
\linkedin{linkedin.com/in/q00}
% \gitlab{gitlab-id}
% \stackoverflow{SO-id}{SO-name}
% \twitter{@twit}
% \skype{skype-id}
% \reddit{reddit-id}
% \medium{madium-id}
% \googlescholar{googlescholar-id}{name-to-display}
%% \firstname and \lastname will be used
% \googlescholar{googlescholar-id}{}
% \extrainfo{extra informations}

\quote{``prioritize efficiency while actively avoiding inaction."}


%-------------------------------------------------------------------------------
\begin{document}

% Print the header with above personal informations
% Give optional argument to change alignment(C: center, L: left, R: right)
\makecvheader[C]

% Print the footer with 3 arguments(<left>, <center>, <right>)
% Leave any of these blank if they are not needed
\makecvfooter
  {\today}
  {JaeGyu Lee~~~·~~~Resume}
  {\thepage}


%-------------------------------------------------------------------------------
%	CV/RESUME CONTENT
%	Each section is imported separately, open each file in turn to modify content
%-------------------------------------------------------------------------------
%-------------------------------------------------------------------------------
%	SECTION TITLE
%-------------------------------------------------------------------------------
\cvsection{Summary}


%-------------------------------------------------------------------------------
%	CONTENT
%-------------------------------------------------------------------------------
\begin{cvparagraph}
Current lead backend engineer at start-up company Onuii. 6+ years experience specializing in the baackend development, infrastructure design, and data management. Always try to prioritize efficiency while actively avoiding inaction. My experience is defined with continuous activity and proactive Approach. Interested in ensuring consistency in distributed environment, and learning new technologies about data science.

\end{cvparagraph}

%-------------------------------------------------------------------------------
%	SECTION TITLE
%-------------------------------------------------------------------------------
\cvsection{Work Experience}


%-------------------------------------------------------------------------------
%	CONTENT
%-------------------------------------------------------------------------------
\begin{cventries}

%---------------------------------------------------------
  \cventry
    {Lead Backend Engineer} % Job title
    {Onuii(설탭)} % Organization
    {Seoul, S.Korea} % Location
    {Mar. 2022 - Now} % Date(s)
    {
      \begin{cvitems} % Description(s) of tasks/responsibilities
        \item {Fostered team growth through one-on-one meetings, resulting in an average skill level increase across 14 team members.}
        \item {Deployed Kakfa echo system with golang, including consumer proxy, producer, monitoring, to establish an event-driven architecture.}
        \item {Implemented a GraphQL gateway for efficient request handling to gRPC services, enhancing endpoint integration and caching. Customizing graphql-mesh, using explicit optional in protobuf3 to handle graphql nullable.}
        \item {Designed Data pipeline to make a data lake using AWS EKS,EMR, Trino, Datahub, Spark.}
        \item {Provisioned microservice infrastructure using AWS ECS, ECR, and Service Connect discovery, employing IaC tools like Packer and Terraform for seamless deployment.}
        \item {Transitioned from monolithic and serverless legacy systems to a microservice architecture using Kafka and gRPC for service discovery.}
        \item {Developed a teacher-student matching algorithm, leveraging AWS Opensearch for data analysis based on sensitive student information.}
        \item {Introduced "Risk Search" as a proactive measure in the development process.}
        \item {Utilized Post-mortem analyses to create a historical knowledge base for domain-related issues.}
      \end{cvitems}
    }

%---------------------------------------------------------
  \cventry
    {CTO} % Job title
    {Flatgarden} % Organization
    {Seoul, S.Korea} % Location
    {Mar. 2020 - Mar. 2022} % Date(s)
    {
      \begin{cvitems} % Description(s) of tasks/responsibilities
        \item {Deployed "입시 고민 메신저 학학이" in App store and google play}
        \item {Built infrastructure utilizing Docker and Terraform with AWS ECS in AWS architecture and migrated legacy to ecs fargate}
        \item {Designed real-time chat bot with Firebase FireStore for scalable service.}
        \item {Implemented modified scrum by using zenhub, notion to effectively communicate}
      \end{cvitems}
    }

%---------------------------------------------------------
  \cventry
    {Graduate Student Researcher} % Job title
    {Seoul National University of Science and Technology} % Organization
    {Seoul, S.Korea} % Location
    {May. 2019 - Feb. 2022} % Date(s)
    {
      \begin{cvitems} % Description(s) of tasks/responsibilities
        \item {Implemented TPC-C Benchmark(WAS, RTE, DB) on IBM DB2 using C, Golang with TTA(Telecommunications Technology Association)}
        \item {Developed an Intelligent (AI) Small Water Utility Integrated Management System with Kafka, Telegraf with Kafka Connect, InfluxDB, Grafana with WonJu-si}
        \item {Implemented a complete and fast Scraping method for collecting tweets with NSR(National Security Researcher Institute) }
      \end{cvitems}
    }

%---------------------------------------------------------
  \cventry
    {Software Engineer} % Job title
    {Hyundai-Medi} % Organization
    {Seoul, S.Korea} % Location
    {DEC. 2018 - MAR. 2019} % Date(s)
    {
      \begin{cvitems} % Description(s) of tasks/responsibilities
        \item {Implemented Hospital and Pharmacy Search service using Python Django rest framework.}
        \item {Crawled medicine information}
      \end{cvitems}
    }

%---------------------------------------------------------
  \cventry
    {Software Engineer} % Job title
    {Playauto} % Organization
    {Seoul, S.Korea} % Location
    {AUG. 2017 - NOV. 2018} % Date(s)
    {
      \begin{cvitems} % Description(s) of tasks/responsibilities
        \item {Implemented data scraping of E-Commerce service like Coupang, G-market for automating registering goods and inventory management }
        \item {Developed an inventory management system for enterprise customer}
      \end{cvitems}
    }

%---------------------------------------------------------
\end{cventries}

%-------------------------------------------------------------------------------
%	SECTION TITLE
%-------------------------------------------------------------------------------
\cvsection{Education}


%-------------------------------------------------------------------------------
%	CONTENT
%-------------------------------------------------------------------------------
\begin{cventries}

  \cventry
    {M.S. with Database Performance Evaluation, Benchmark} % Degree
    {Seoul National University of Science and Technology} % Institution
    {Nowon, S.Korea} % Location
    {Sep. 2021 - Aug. 2023} % Date(s)
    {
      \begin{cvitems} % Description(s) bullet points
        \item {Big Data Management and Application Lab, SeoulTech}
        \item {Research: Concurrency Control, Distributed transaction, Kafka}
      \end{cvitems}
    }

%---------------------------------------------------------
  \cventry
    {B.S. in Computer Science and Engineering} % Degree
    {Seoul National University of Science and Technology} % Institution
    {Nowon, S.Korea} % Location
    {Mar. 2019 - Aug. 2021} % Date(s)
    {
      \begin{cvitems} % Description(s) bullet points
      \end{cvitems}
    }

%---------------------------------------------------------
\end{cventries}

%-------------------------------------------------------------------------------
%	SECTION TITLE
%-------------------------------------------------------------------------------
\cvsection{Papers}


%-------------------------------------------------------------------------------
%	SUBSECTION TITLE
%-------------------------------------------------------------------------------


%-------------------------------------------------------------------------------
%	CONTENT
%-------------------------------------------------------------------------------
\begin{cvhonors}

%---------------------------------------------------------
  \cvhonor
    {Supporting Distributed Transactions for Elasticsearch based on Two-Phase Commit Dwell Lock} % Award
    {Graduate thesis - Seoul National University of Science and TechNology} % Event
    {Seoul, S.Korea} % Location
    {2023} % Date(s)

%---------------------------------------------------------
\end{cvhonors}


%-------------------------------------------------------------------------------
%	SUBSECTION TITLE
%-------------------------------------------------------------------------------


%-------------------------------------------------------------------------------
%	CONTENT
%-------------------------------------------------------------------------------
\begin{cvhonors}
%---------------------------------------------------------
  \cvhonor
    {Data Pipeline for a Real-Time Water Management System with Time Series Prediction Model} % Award
    {ICESI2022 (International Conference on Electric Vehicle, Smart Grid, and Information Technology 2022)} % Event
    {Seoul, S.Korea} % Location
    {2022} % Date(s)

%---------------------------------------------------------
  \cvhonor
    {TPC-C Benchmarking for ElasticSearch} % Award
    {International Conference on Big Data and Smart Computing (BIGCOMP)} % Event
    {Seoul, S.Korea} % Location
    {2022} % Date(s)

%---------------------------------------------------------
  \cvhonor
    {A Complete and Fast Scraping Method for Collecting Tweets} % Award
    {International Conference on Big Data and Smart Computing (BIGCOMP)} % Event
    {Seoul, S.Korea} % Location
    {2021} % Date(s)


%---------------------------------------------------------
\end{cvhonors}

%-------------------------------------------------------------------------------
%	SECTION TITLE
%-------------------------------------------------------------------------------
\cvsection{Presentation}


%-------------------------------------------------------------------------------
%	CONTENT
%-------------------------------------------------------------------------------
\begin{cventries}

%---------------------------------------------------------
  \cventry
    {AWS Korea Edutech Community Forum 2023} % Role
    {Case studies of building MSA and BI in a distributed environment through AWS ECS, EKS, and EMR} % Event
    {Seoul, S.Korea} % Location
    {June. 2023} % Date(s)
    {
      \begin{cvitems} % Description(s)
        \item {\href{https://pages.awscloud.com/edutech-community-forum-2023.html}{Edutech Community Form 2023 Detail Page}}
        \item {\href{https://www.linkedin.com/feed/update/urn:li:activity:7082573017570021376/}{Detail Description in korean}}
        \item {\href{https://www.slideshare.net/JQLEE6/aws-korea-edutech-community-forum-2023-pdf}{Presentation Keynote}}
      \end{cvitems}
    }

%---------------------------------------------------------
  \cventry
    {DevFest 2023 Golang Korea - Go to Daegu} % Role
    {Implementing a Simple and Easy Chat System in Golang with AWS DynamoDB} % Event
    {Daegu, S.Korea} % Location
    {DEC. 2023} % Date(s)
    {
      \begin{cvitems} % Description(s)
        \item {Hands on about chat system using channel and graphql subscription}
        \item {\href{https://festa.io/events/4339}{DevFest 2023 Golang Korea Detail Page}}
      \end{cvitems}
    }

%---------------------------------------------------------
\end{cventries}

%-------------------------------------------------------------------------------
%	SECTION TITLE
%-------------------------------------------------------------------------------
\cvsection{Extracurricular Activity}


%-------------------------------------------------------------------------------
%	CONTENT
%-------------------------------------------------------------------------------
\begin{cventries}
  \cventry
    {Mentor} % Affiliation/role
    {Prography - IT Community} % Organization/group
    {Seoul, S.Korea} % Location
    {Aug. 2018 - FEB. 2020} % Date(s)
    {
      \begin{cvitems} % Description(s) of experience/contributions/knowledge
        \item {\href{https://prography.org/}{https://prography.org/}}
        \item {Nodejs Mentor}
      \end{cvitems}
    }
    
  \cventry
    {Leader} % Affiliation/role
    {Roubit - routine managing system} % Organization/group
    {Seoul, S.Korea} % Location
    {Mar. 2019 - JAN. 2020} % Date(s)
    {
      \begin{cvitems} % Description(s) of experience/contributions/knowledge
        \item {\href{https://roubit.me/}{Service homepage}}
        \item {Implemented routine managing service with NFC tags.}
      \end{cvitems}
    }

%---------------------------------------------------------
  \cventry
    {Backend Engineer} % Affiliation/role
    {Roubit - Routine managing system} % Organization/group
    {Seoul, S.Korea} % Location
    {Mar. 2019 - JAN. 2020} % Date(s)
    {
      \begin{cvitems} % Description(s) of experience/contributions/knowledge
        \item {\href{https://roubit.me/}{Service homepage}}
        \item {Implemented routine managing service with NFC tags.}
      \end{cvitems}
    }

%---------------------------------------------------------
  \cventry
    {Backend Engineer} % Affiliation/role
    {PPLE - Blood donation project} % Organization/group
    {Seoul, S.Korea} % Location
    {Mar. 2019 - JAN. 2020} % Date(s)
    {
      \begin{cvitems} % Description(s) of experience/contributions/knowledge
        \item {\href{https://pple.link/}{Service homepage}}
        \item {Implemented Direct Blood Donation Process on service.}
      \end{cvitems}
    }

%---------------------------------------------------------
\end{cventries}



%-------------------------------------------------------------------------------
\end{document}
